%% This file is designed to be included into a LaTeX document
%% See http://www.latex-project.org/ for information about LaTeX
%% biogeme 2.4
%% Compiled Sun Oct 25 23:01:47 WEST 2015
%% Michel Bierlaire, EPFL
%% Report created on 07/26/16 10:39:52
Nothing here\\


\begin{flushleft}
\begin{tabular}{rcl}
\hline
Model &:& Logit\\
Number of estimated parameters&:&6\\
Number of  observations &:& 838\\
Number of individuals&:&838\\
Null log likelihood&:&-985.941\\
Init log likelihood&:&-985.941\\
Final log likelihood&:&-587.394\\
Likelihood ratio test &:&797.095\\
Rho-square&:&0.404\\
Adjusted rho-square&:&0.398\\
Final gradient norm&:&+1.951e-004\\
Diagnostic&:&Convergence reached...\\
Iterations&:&14\\
Run time&:&00:00\\
Variance-covariance&:&from analytical hessian
\\
Sample file&:&models-data/data_26jul_1.dat\\
\end{tabular}
\end{flushleft}
%%
%%
%%
  \begin{tabular}{l}
\begin{tabular}{rlr@{.}lr@{.}lr@{.}lr@{.}l}
         &                       &   \multicolumn{2}{l}{}    & \multicolumn{2}{l}{Robust}  &     \multicolumn{4}{l}{}   \\
Parameter &                       &   \multicolumn{2}{l}{Coeff.}      & \multicolumn{2}{l}{Asympt.}  &     \multicolumn{4}{l}{}   \\
number &  Description                     &   \multicolumn{2}{l}{estimate}      & \multicolumn{2}{l}{std. error}  &   \multicolumn{2}{l}{$t$-stat}  &   \multicolumn{2}{l}{$p$-value}   \\

\hline

1 & ASC\_FREE & 14&6 & 0&126 & 115&60 & 0&00 \\
2 & ASC\_OTHER & -1&01 & 0&155 & -6&55 & 0&00 \\
3 & ASC\_TRANSIT & -1&33 & 0&134 & -9&91 & 0&00 \\
4 & B\_BIZ & 0&490 & 0&245 & 2&00 & 0&05 \\
5 & B\_HYP & 0&0798 & 0&0240 & 3&32 & 0&00 \\
6 & B\_VIS & -0&418 & 0&194 & -2&16 & 0&03 \\
\hline

\end{tabular}
\\
\begin{tabular}{rcl}
\multicolumn{3}{l}{\bf Summary statistics}\\
\multicolumn{3}{l}{ Number of observations = $838$} \\
 $\mathcal{L}(0)$ &=&  $-985.941$ \\
 $\mathcal{L}(c)$ &=& ???\\
 $\mathcal{L}(\hat{\beta})$ &=& $-587.394 $  \\
 $-2[\mathcal{L}(0) -\mathcal{L}(\hat{\beta})]$ &=& $797.095$ \\
    $\rho^2$ &=&   $0.404$ \\
    $\bar{\rho}^2$ &=&    $0.398$ \\
\end{tabular}
\end{tabular}
\end{document}
%%% Another joint format
%%%
